
\documentclass[12pt]{article}
\usepackage{amsmath,amsfonts,amssymb}
\usepackage{graphicx}
\usepackage{hyperref}
\usepackage{geometry}
\geometry{margin=1in}

\title{The Hamsadhwani Group: A Universal Generative Framework for Recursion, Constants, and Conscious Interface}
\author{Author Name}
\date{\today}

\begin{document}

\maketitle

\begin{abstract}
We propose the \textbf{Hamsadhwani Group (HS-Group)} as a minimal, generative algebra composed of six elements --- \{e, $\pi$, i, 1, 0, $\chi$\} --- that together define a universal recursive interface. This group emerges at the convergence of mathematical structure, physical law, biological recursion, musical symmetry, and the act of observation itself.

Each element plays a symbolic and functional role: growth, curvature, spin, identity, vacuum, and chirality. The group offers a conceptual bridge between quantum field theory, category theory, biological recursion, type systems, and metaphysical consciousness.

\textit{In science as in poetry, just the sufficient and the necessary --- fading in and away.}
\end{abstract}

\tableofcontents
\newpage

\section{Introduction}
Context, foundational question, recursion across domains. Introduce Aspinīya and the goal of unifying structure via HS-Group.

\section{The Hamsadhwani Group}
\subsection{Definition}
\subsection{Properties}
\subsection{Interpretations: Algebraic, Typing, Symbolic}

\section{Recursive Interfaces and Type Systems}
\subsection{Typing as Observation}
\subsection{Category Theory and Functional Programming}
\subsection{The Role of Chirality}

\section{Domain Mappings}
\subsection{Physics: Constants and Fields}
\subsection{Biology: Carbon and Mutation}
\subsection{Mathematics and Computation}

\section{The Poetic and the Metaphysical}
\subsection{Upaniṣadic Echoes}
\subsection{Observerhood and Conscious Recursion}
\subsection{The Tuning of the Universe}

\section{Figures and Diagrams}
\subsection{Symbolic Representations}
\subsection{Loop Structures}
\subsection{Interface Glyphs}

\section{Related Work}
Foundational references in physics, logic, CS, philosophy, music theory.

\section{Future Work}
Speculations: string theory, GPT recursion, ritual modeling, recursive art.

\section{Conclusion}
Reassert the invocation. Call for contributors. Recursive silence.

\appendix
\section{Symbol Table}
\section{Code Samples}
\section{Mathematical Proof Sketches}

\end{document}
